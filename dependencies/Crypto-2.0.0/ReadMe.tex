\documentclass{article}

\usepackage{listings}
\usepackage{a4}
\usepackage{courier}
\usepackage{hyperref}
\usepackage{html}


\lstdefinelanguage{ASN1} {
  morekeywords={},
  sensitive=false,
  morecomment=[s]{(--}{--)}
  }

\lstdefinelanguage{shell} {
  sensitive=true
  }

\setlength{\parskip}{\medskipamount}
\setlength{\parindent}{0pt}

\title{Haskell Cryptographic Library 2.0.1}
\author{Dominic Steinitz}

\begin{document}

\maketitle

The 
\htmladdnormallinkfoot
{Haskell Cryptographic Library 2.0.1}
{http://www.haskell.org/crypto}
collects together existing Haskell cryptographic
functions into one cabalized package, together with HUnit tests,
QuickCheck property tests, examples showing how to interwork with
other cryptographic implementations and examples showing how to 
handle other ASN.1 definitions.

This release contains:
\begin{itemize}
\item DES
\item Blowfish
\item AES
\item Cipher Block Chaining (CBC)
\item PKCS\#5 and nulls padding
\item SHA-1
\item MD5
\item RSA
\item OAEP
\item ASN.1
\item PKCS\#8
\item X.509
\end{itemize}

Haddock documentation for the library is available
\htmladdnormallinkfoot
{here}
{http://www.haskell.org/crypto/doc/html}
.

This document was last updated on 24th April 2005.
\copyright\ 2005 Dominic Steinitz. 

\section{System Requirements}

\begin{itemize}
\item
The code has been tested on GHC 6.4.
It does not currently work with Hugs or NHC. 
\item
It requires the use of
\htmladdnormallinkfoot
{{\tt NewBinary.Binary}}
{http://www.n-heptane.com/nhlab/repos/NewBinary}
.
This uses an old version of cabal and you will need to download 
\htmladdnormallinkfoot
{this}
{http://www.haskell.org/crypto/downloads/NewBinary.cabal}
this
in order to build and install it.
\end{itemize}

\section{Installation Instructions}

Get the sources:

\lstset{language=shell,basicstyle=\ttfamily\small}
\begin{lstlisting}[frame=single]
darcs get --tag "2.0.1" http://www.haskell.org/crypto/src
\end{lstlisting}

Build and install ready for testing:

\begin{lstlisting}[frame=single]
ghc -o Setup Setup.hs -package Cabal
./Setup configure --prefix=/my/chosen/dir
./Setup build
./Setup install --user
\end{lstlisting}

Run the tests.

\begin{lstlisting}[frame=single]
cd /my/chosen/dir/bin
./BERTest
./RSATest
./SymmetricTest
./QuickTest
\end{lstlisting}

Run the examples. For this you will need to generate an X.509 certificate
and a PKCS\#8 file containing a private key. Here's one way to do this.
Be careful as not all OIDs are supported (although they are easy to add).

Notes:
\begin{itemize}
\item
The private key is produced in PKCS\#8 format.
You can look at it with Peter Gutmann's ASN.1 
\htmladdnormallinkfoot
{object dump program}
{http://www.cs.auckland.ac.nz/~pgut001/dumpasn1.c}
if required. 
\item
The private key is unencrypted. Storing private keys in clear 
is not good security practice. The current implementation does not support 
encrypted private keys but this should be straightforward to add.
\end{itemize}

\begin{lstlisting}[frame=single]
openssl req -new -nodes -x509 -outform DER -out foo.der \
-keyout key.pem -days 365 \
-subj "/C=US/L=City/CN=localhost/emailAddress=root@localhost"

openssl pkcs8 -topk8 -inform PEM -outform DER -in key.pem \
-nocrypt -out key.der
\end{lstlisting}

Run the examples to examine your private and public keys.

\begin{lstlisting}[frame=single]
./PKCS8Example key.der
./X509Example foo.der
\end{lstlisting}

Run the example to encrypt using Haskell and decrypt using openssl.

\begin{lstlisting}[frame=single]
echo plaintext > plaintext

./RSAEncryptionExample --cert=foo.der --plain=plaintext \
--cipher=ciphertext

openssl rsautl -decrypt -inkey key.pem -in ciphertext \ 
-out unciphertext -oaep
\end{lstlisting}

When you are happy, build and install them in
their final destination:

\begin{lstlisting}[frame=single]
./Setup unregister --user
./Setup clean
./Setup configure
./Setup build
./Setup install
\end{lstlisting}

\section{ASN.1 Support}

The package contains some enough general ASN.1 support to decode PKCS\#8 and
X.509 encoded using BER. See the examples that
are provided with the package: \begin{tt}X509Example\end{tt} and
\begin{tt}PKCS8Example\end{tt}.

Here's a further example.

Suppose you have an ASN.1 module and have checked that it conforms to
the standard using, for example, 
the on-line tool, 
\htmladdnormallinkfoot{Asnp}{http://asn1.elibel.tm.fr/en/tools/asnp/index.htm}
\footnote{Definitions using ANY DEFINED BY have to be checked with -1990.}
\footnote{Asnp was developed in Objective Caml.}

\lstset{language=ASN1}
\begin{lstlisting}[frame=single]
FooBar {1 2 0 0 6 1} DEFINITIONS ::= 
   BEGIN 
      Journey ::= SEQUENCE {
         origin IA5String,
         stop1 [0] IA5String OPTIONAL,
         stop2 [1] IA5String OPTIONAL,
         destination IA5String
      }
      Odyssey ::= SEQUENCE {
         start Journey,
         trip1 [0] Journey OPTIONAL,
         trip2 [1] Journey OPTIONAL,
         trip3 [2] Journey OPTIONAL,
         end Journey
      }
   END 
\end{lstlisting}

Create abstract Haskell representations of the ASN.1 types.

\lstset{language=Haskell}
\begin{lstlisting}[frame=single]
journey =
   "Journey" ::=
      AbsSeq Universal 16 Implicit [
         Regular  (Just "origin"       :>: 
            (Nothing  :@: absIA5String)),
         Optional (Just "stop1"        :>: 
            (Just 0   :@: absIA5String)),
         Optional (Just "stop2"        :>: 
            (Just 1   :@: absIA5String)),
         Regular  (Just "destination"  :>: 
            (Nothing  :@: absIA5String))
      ]

odyssey =
   "Odyssey" ::=
      AbsSeq Universal 16 Implicit [
         Regular  (Just "start"       :>: (Nothing  :@: journey)),
         Optional (Just "trip1"       :>: (Just 0   :@: journey)),
         Optional (Just "trip2"       :>: (Just 1   :@: journey)),
         Optional (Just "trip3"       :>: (Just 2   :@: journey)),
         Regular  (Just "end"         :>: (Nothing  :@: journey))
      ]
\end{lstlisting}

Then you can check that (abstract representations of) BER values conform
to these types.

\begin{lstlisting}[frame=single]
module Main(main) where

import Codec.ASN1.BER
import Codec.ASN1

-- Other definitions

main = 
   do w <- tc odyssey berValue
      putStrLn (show w)
\end{lstlisting}

Now suppose that the BER value conforms to the ASN.1 type and that
you wish to manipulate the various values in the encoding. First
create some Haskell types to hold the decoded values.

\begin{lstlisting}[frame=single]
data Journey =
   Journey {
      origin :: IA5String,
      stop1 :: Maybe IA5String,
      stop2 :: Maybe IA5String,
      destination :: IA5String
   }
   deriving (Eq,Show)

data Odyssey =
   Odyssey {
      start :: Journey,
      trip1 :: Maybe Journey,
      trip2 :: Maybe Journey,
      trip3 :: Maybe Journey,
      end   :: Journey
   }
   deriving (Eq,Show)
\end{lstlisting}

Then make them instances of the class {\begin{tt}Encode\end{tt}}.

\begin{lstlisting}[frame=single]
instance Encode Journey where
   decode a b =
      do x <- b
         let as = absSeqComponents a
             bs  = encodedDefComps x
             as' = map (\x -> replaceRef x as bs) as
         return (
            Journey {
               origin      = fromJust (decode (as'!!0) (bs!!0)),
               stop1       = decode (as'!!1) (bs!!1),
               stop2       = decode (as'!!2) (bs!!2),
               destination = fromJust (decode (as'!!3) (bs!!3))            
            }
         )

instance Encode Odyssey where
   decode a b =
      do x <- b
         let as = absSeqComponents a
             bs  = encodedDefComps x
             as' = map (\x -> replaceRef x as bs) as
         return (
            Odyssey {
               start  = fromJust (decode (as'!!0) (bs!!0)),
               trip1  = decode (as'!!1) (bs!!1),
               trip2  = decode (as'!!2) (bs!!2),
               trip3  = decode (as'!!2) (bs!!2),
               end    = fromJust (decode (as'!!4) (bs!!4))
            }
         )
\end{lstlisting}

Finally, you can extract values from the decoded value.

\begin{lstlisting}[frame=single]
module Main(main) where

import Codec.ASN1.BER
import Codec.ASN1

-- Other definitions

main = 
   do w <- tc odyssey berValue
      let x = fromJust (decode (getAbsType odyssey) (Just w))
      putStrLn (show (start x))
\end{lstlisting}

\section{To Do}

In no particular order:

\begin{itemize}
\item
Read and generate PKCS12 private keys so that it is easy to inter-work 
with other RSA implementations.
\item
Incorporate other symmetric key algorithms already coded in Haskell.
\item
Performance analysis as Blowfish ought to run more quickly than DES.
\item
Other modes / padding schemes.
\item
Digital signatures.
\item
Improve the support for {\begin{tt}ANY DEFINED BY\end{tt}} and its replacement
in later versions of the ASN.1 standards. In particular, replace the index
number by a label and do not treat as equivalent to {\tt OPTIONAL} or
{\tt DEFAULT}.
\item
Extend typechecking to ensure that only the appropriate key sizes are used
for a given algorithm.
\item
Put ASN.1 definitions into more appropriate modules, for example,
{\tt AlgorithmIdentifier}.
\item
Check lengths for correctness when decoding BER. See the Codec.ASN1.TLV
source for more on this.
\item
Improve the error messages when checking a BER encoding against its ASN.1
specification. Currently the messages say what has failed but does not
give the context.
\end{itemize}

\section{Contact}

All questions, comments, bug reports, flames, requests for 
updates / changes and suggestions should be directed to Dominic Steinitz.

\section{Licensing}

The modules in the library come from different authors and have been 
released under different licences. 

\subsection{Contributors}

\subsubsection{Codec.ASN1}
\begin{tabular}{|p{6cm}|p{3cm}|p{3cm}|p{1cm}|}
\hline\hline
Codec.ASN1 & Dominic Steinitz & 
Copyright \copyright\ 2005, All rights reserved & BSD \\
\hline
Codec.ASN1.BER & Dominic Steinitz & 
Copyright \copyright\ 2005, All rights reserved & BSD \\
\hline
Codec.ASN1.PKCS8 & Dominic Steinitz & 
Copyright \copyright\ 2005, All rights reserved & BSD \\
\hline
Codec.ASN1.TLV & Dominic Steinitz & 
Copyright \copyright\ 2005, All rights reserved & BSD \\
\hline
Codec.ASN1.X509 & Dominic Steinitz & 
Copyright \copyright\ 2005, All rights reserved & BSD \\
\hline\hline
\end{tabular}

\subsubsection{Codec.Binary}
\begin{tabular}{|p{6cm}|p{3cm}|p{3cm}|p{1cm}|}
\hline\hline
Codec.Binary.Base64 & Warrick Gray & 
Copyright \copyright\ 2005, All rights reserved & BSD \\
\hline\hline
\end{tabular}

\subsubsection{Codec.Encryption}
\begin{tabular}{|p{6cm}|p{3cm}|p{3cm}|p{1cm}|}
\hline\hline
Codec.Encryption.AES & Lukasz Anforowicz &
Copyright \copyright\ 2005, All rights reserved & BSD \\
\hline
Codec.Encryption.AESAux & Dominic Steinitz &
Copyright \copyright\ 2005, All rights reserved & BSD \\
\hline
Codec.Encryption.Blowfish & Doug Hoyte &
Copyright \copyright\ 2005, All rights reserved & BSD \\
\hline
Codec.Encryption.BlowfishAux & Dominic Steinitz &
Copyright \copyright\ 2005, All rights reserved & BSD \\
\hline
Codec.Encryption.DES & Ian Lynagh &
Copyright \copyright\ 2005, All rights reserved & BSD \\
\hline
Codec.Encryption.DESAux & Dominic Steinitz &
Copyright \copyright\ 2005, All rights reserved & BSD \\
\hline
Codec.Encryption.Modes & Dominic Steinitz &
Copyright \copyright\ 2005, All rights reserved & BSD \\
\hline
Codec.Encryption.Padding & Dominic Steinitz &
Copyright \copyright\ 2005, All rights reserved & BSD \\
\hline
Codec.Encryption.RSA & David Sankel &
Copyright \copyright\ 2005, All rights reserved & GPL \\
\hline
Codec.Encryption.RSA.EMEOAEP & David Sankel &
Copyright \copyright\ 2005, All rights reserved & GPL \\
\hline
Codec.Encryption.RSA.MGF & Dominic Steinitz &
Copyright \copyright\ 2005, All rights reserved & BSD \\
\hline
Codec.Encryption.RSA.NumberTheory & David Sankel &
Copyright \copyright\ 2005, All rights reserved & GPL \\
\hline\hline
\end{tabular}

\subsubsection{Codec}
\begin{tabular}{|p{6cm}|p{3cm}|p{3cm}|p{1cm}|}
\hline\hline
Codec.Utils & Dominic Steinitz &
Copyright \copyright\ 2005, All rights reserved & BSD \\
\hline\hline
\end{tabular}

\subsubsection{Data.Digest}
\begin{tabular}{|p{6cm}|p{3cm}|p{3cm}|p{1cm}|}
\hline\hline
Data.Digest.MD5 & Dominic Steinitz &
Copyright \copyright\ 2005, All rights reserved & BSD \\
\hline
Data.Digest.MD5Aux & Ian Lynagh &
Copyright \copyright\ 2005, All rights reserved & BSD \\
\hline
Data.Digest.SHA1 & Dominic Steinitz &
Copyright \copyright\ 2005, All rights reserved & BSD \\
\hline
Data.Digest.SHA1Aux & Ian Lynagh &
Copyright \copyright\ 2005, All rights reserved & BSD \\
\hline\hline
\end{tabular}

\subsubsection{Data}
\begin{tabular}{|p{6cm}|p{3cm}|p{3cm}|p{1cm}|}
\hline\hline
Data.LargeWord 
& Dominic Steinitz &
Copyright \copyright\ 2005, All rights reserved & BSD \\
\hline\hline
\end{tabular}

\subsubsection{Tests and Examples}
\begin{tabular}{|p{6cm}|p{3cm}|p{3cm}|p{1cm}|}
\hline\hline
BERTest
& Dominic Steinitz &
Copyright \copyright\ 2005, All rights reserved & BSD \\
\hline
RSATest
& Dominic Steinitz &
Copyright \copyright\ 2005, All rights reserved & BSD \\
\hline
X509Example
& Dominic Steinitz &
Copyright \copyright\ 2005, All rights reserved & BSD \\
\hline
QuickTest
& Dominic Steinitz &
Copyright \copyright\ 2005, All rights reserved & BSD \\
\hline
SymmetricTest
& Dominic Steinitz &
Copyright \copyright\ 2005, All rights reserved & BSD \\
\hline
PKCS8Example
& Dominic Steinitz &
Copyright \copyright\ 2005, All rights reserved & BSD \\
\hline
RSAEncryptionExample
& Dominic Steinitz &
Copyright \copyright\ 2005, All rights reserved & BSD \\
\hline\hline
\end{tabular}

\subsection{The BSD License}

This license is based on
\htmladdnormallinkfoot
{The BSD License}
{http://www.opensource.org/licenses/bsd-license.php}.

Redistribution and use in source and binary forms, with or without 
modification, are permitted provided that the following conditions are met:

\begin{itemize}
\item
Redistributions of source code must retain the above copyright notice, 
this list of conditions and the following disclaimer.
\item
Redistributions in binary form must reproduce the above copyright notice, 
this list of conditions and the following disclaimer in the documentation 
and/or other materials provided with the distribution.
\item
The names of its contributors may not be used to endorse or promote 
products derived from this software without specific prior written permission.
\end{itemize}

\begin{sc}
This software is provided by the copyright holders and contributors ``AS IS'' 
and any express or implied warranties, including, but not limited to, 
the implied warranties of merchantability and fitness for a particular 
purpose are disclaimed. In no event shall the copyright onwers or
contributors be liable for any direct, indirect, incidental, special,
exemplary, or consequential damages (including, but not limited to,
procurement of substitute goods or services; loss of use, data, or profits;
or business interruption) however caused and on any theory of liability,
whether in contract, strict liability, or tort (including negligence or
otherwise) arising in any way out of the use of this software,
even if advised of the possibility of such damage.
\end{sc}

\subsection{The GNU General Public License (GPL)}

This program is free software; you can redistribute it and/or modify it under the terms of the GNU General Public License as published by the Free Software Foundation; either version 2 of the License, or (at your option) any later version.

This program is distributed in the hope that it will be useful, but WITHOUT ANY WARRANTY; without even the implied warranty of MERCHANTABILITY or FITNESS FOR A PARTICULAR PURPOSE. See the GNU General Public License for more details.

You can find a copy of
the GNU General Public License 
\htmladdnormallinkfoot
{here}
{http://www.opensource.org/licenses/gpl-license.php}
; 
if not, write to the Free Software Foundation, Inc., 59 Temple Place, 
Suite 330, Boston, MA 02111-1307 USA

\section{Disclaimer}
Cryptography is a notoriously easy area in which to make mistakes, 
not necessarily with the algorithms but with how they are implemented 
(for example not protecting keys, using weak keys and so on). 
For a readable account of some of the pitfalls, see 
\htmladdnormallinkfoot
{Ross Anderson}
{http://www.cl.cam.ac.uk/users/rja14/}
's book.

\begin{sc}
This software is provided by the copyright holders and contributors ``AS IS'' 
and any express or implied warranties, including, but not limited to, 
the implied warranties of merchantability and fitness for a particular 
purpose are disclaimed. In no event shall the copyright onwers or
contributors be liable for any direct, indirect, incidental, special,
exemplary, or consequential damages (including, but not limited to,
procurement of substitute goods or services; loss of use, data, or profits;
or business interruption) however caused and on any theory of liability,
whether in contract, strict liability, or tort (including negligence or
otherwise) arising in any way out of the use of this software,
even if advised of the possibility of such damage.
\end{sc}

\section{Acknowledgements}

\begin{itemize}
\item
Doug Hoyte (HardCore SoftWare)
\item
\htmladdnormallinkfoot
   {Ian Lynagh}
   {http://web.comlab.ox.ac.uk/oucl/work/ian.lynagh}
\item
\htmladdnormallinkfoot
   {David Sankel}
   {http://www.electronconsulting.com/whois.html}
\item
\htmladdnormallinkfoot 
{Ross Paterson}
{http://www.soi.city.ac.uk/~ross}
\item
Lukasz Anforowicz
\item
\htmladdnormallinkfoot 
{Warrick Gray}
{http://homepages.paradise.net.nz/warrickg/haskell/http/}
\end{itemize}

\end{document}
